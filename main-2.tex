\documentclass[12pt,letterpaper]{article}
\usepackage[spanish,es-nodecimaldot]{babel}
\usepackage[utf8]{inputenc}
\usepackage[T1]{fontenc}
\usepackage{float}
\usepackage{bussproofs}

%% Sets page size and margins
\usepackage[a4paper,top=2.5cm,bottom=2.5cm,left=2cm,right=2cm]{geometry}

\usepackage{amsmath}
\usepackage{amssymb}
\usepackage{amsthm}
\usepackage{mathtools}
\usepackage{listings}
\usepackage{enumitem}[shortlabels]
\usepackage[colorinlistoftodos]{todonotes}
\usepackage[colorlinks=true, allcolors=blue]{hyperref}
\usepackage{url}

\usepackage{lipsum}

\usepackage{tcolorbox}

%Author affil
\usepackage{authblk}

%% Title
\title{
		\vspace{-0.7in} 	
		\usefont{OT1}{bch}{b}{n}
		
		\begin{minipage}{3cm}
        \vspace{-0.5in} 	
    	\begin{center}
    		
    	\end{center}
    \end{minipage}\hfill
    \begin{minipage}{10.7cm}
    	\begin{center}

\normalfont \normalsize \textsc{UNIVERSIDAD NACIONAL AUTÓNOMA DE MÉXICO \\ FACULTAD DE CIENCIAS \\ LENGUAJES DE PROGRAMACIÓN } \\
		\huge Tarea 03

    	\end{center}
     
    \end{minipage}\hfill
    \begin{minipage}{3.2cm}
    \vspace{-0.5in} 
    	\begin{center}
    		
    	\end{center}
    \end{minipage}

\author{}
\date{}

}




\hypersetup{
    colorlinks=true,
    linkcolor=blue,
    filecolor=magenta,      
    urlcolor=cyan
    }


%------Vectores Unitarios-------%
\newcommand{\uveci}{\hat{\textbf{i}}} 
\newcommand{\uvecj}{\hat{\textbf{j}}}
\newcommand{\uveck}{\hat{\textbf{k}}}

\newcommand{\grade}{^{\circ}}
 \setlength {\marginparwidth }{2cm}
 
\begin{document}

\maketitle

\textbf{Condiciones de entrega:}

\begin{enumerate}
    \item La tarea se realizará en equipos de 3
    \begin{itemize}
        \item Castillo Chora Paola
        \item Ángel Moises González Corrales
        \item López García Luis Norberto
    \end{itemize}
    \item Se debe entregar en un pdf generado con \LaTeX
\end{enumerate}

\vspace{1cm}


Dadas las siguientes expresiones en sintaxis concreta de nuestro lenguaje MiniLisp:

\begin{enumerate}
    \item \texttt{(- (+ 20 3) (- -18 (+ 50 20)))}
    \item \texttt{(not (+ 1 (- 3 (+ -8 1))))}
    \item \texttt{(not (not (+ 3 5)))}
\end{enumerate}

\subsection*{Expresión 1: \texttt{(- (+ 20 3) (- -18 (+ 50 20)))}}

\subsubsection*{(a) Sintaxis Abstracta}
\[ \text{Sintaxis concreta: } \texttt{(- (+ 20 3) (- -18 (+ 50 20)))} \]
\[ \text{Sintaxis abstracta: } \texttt{Sub(Add(Num(20), Num(3)), Sub(Num(-18), Add(Num(50), Num(20))))} \]

\subsubsection*{(b) Evaluación con Semántica Natural}

Evaluamos el \(\texttt{Add(Num(20), Num(3))}\):
\[
\begin{array}{c}
\\
\underline{\texttt{Num(20)} \Rightarrow \texttt{Num(20)}}
\underline{\texttt{Num(3)} \Rightarrow \texttt{Num(3)}} \\
\texttt{Add(Num(20), Num(3))} \Rightarrow \texttt{Num(23)}
\end{array}
\]

Evaluamos \(\texttt{Add(Num(50), Num(20))}\):
\[
\begin{array}{c}
\\
\underline{\texttt{Num(50)} \Rightarrow \texttt{Num(50)}} 
\underline{\texttt{Num(20)} \Rightarrow \texttt{Num(20)}} \\
\texttt{Add(Num(50), Num(20))} \Rightarrow \texttt{Num(70)}
\end{array}
\]

Evaluamos \(\texttt{Sub(Num(-18), Num(70))}\):
\[
\begin{array}{c}
\\
\underline{\texttt{Num(-18)} \Rightarrow \texttt{Num(-18)}}
\underline{\texttt{Num(70)} \Rightarrow \texttt{Num(70)}} \\
\texttt{Sub(Num(-18), Num(70))} \Rightarrow \texttt{Num(-88)}
\end{array}
\]

Evaluamos \(\texttt{Sub(Num(23), Num(-88))}\):
\[
\begin{array}{c}
\\
\underline{\texttt{Num(23)} \Rightarrow \texttt{Num(23)}}
\underline{\texttt{Num(-88)} \Rightarrow \texttt{Num(-88)}} \\
\texttt{Sub(Num(23), Num(-88))} \Rightarrow \texttt{Num(111)}
\end{array}
\]

\textbf{Resultado Final:} \(\texttt{Num(111)}\)

\subsubsection*{(c) Evaluación con Semántica Estructural}

\[
\begin{array}{c}
\text{La expresion es: } \texttt{Sub(Add(Num(20), Num(3)), Sub(Num(-18), Add(Num(50), Num(20))))}. \\
\\
\Rightarrow \texttt{Sub(Num(23), Sub(Num(-18), Add(Num(50), Num(20))))} \\
\Rightarrow \texttt{Sub(Num(23), Sub(Num(-18), Num(70)))} \\
\Rightarrow \texttt{Sub(Num(23), Num(-88)))} \\
\Rightarrow \texttt{Num(111)}

\end{array}
\]

\textbf{Resultado Final:} \(\texttt{Num(111)}\)

\subsection*{Expresión 2: \texttt{(not (+ 1 (- 3 (+ -8 1))))}}

\subsubsection*{(a) Sintaxis Abstracta}
\[
\text{Sintaxis concreta: } \texttt{(not (+ 1 (- 3 (+ -8 1))))}
\]
\[
\text{Sintaxis abstracta: } \texttt{Not(Add(Num(1), Sub(Num(3), Add(Num(-8), Num(1)))))}
\]

\subsubsection*{(b) Evaluación con Semántica Natural}

Evaluamos el \(\texttt{Add(Num(-8), Num(1))}\):
\[
\begin{array}{c}
\\
\underline{\texttt{Num(-8)} \Rightarrow \texttt{Num(-8)}}
\underline{\texttt{Num(1)} \Rightarrow \texttt{Num(1)}} \\
\texttt{Add(Num(-8), Num(1))} \Rightarrow \texttt{Num(-7)}
\end{array}
\]

Evaluamos \(\texttt{Sub(Num(3), Num(-7))}\):
\[
\begin{array}{c}
\\
\underline{\texttt{Num(3)} \Rightarrow \texttt{Num(3)}} 
\underline{\texttt{Num(-7)} \Rightarrow \texttt{Num(-7)}} \\
\texttt{Sub(Num(3), Num(-7))} \Rightarrow \texttt{Num(10)}
\end{array}
\]

Evaluamos \(\texttt{Add(Num(1), Num(10))}\):
\[
\begin{array}{c}
\\
\underline{\texttt{Num(1)} \Rightarrow \texttt{Num(1)}}
\underline{\texttt{Num(10)} \Rightarrow \texttt{Num(10)}} \\
\texttt{Add(Num(1), Num(10))} \Rightarrow \texttt{Num(11)}
\end{array}
\]

Evaluamos \(\texttt{Not(Num(11))}\):
\[
\begin{array}{c}
\\
\underline{\texttt{Num(11)} \Rightarrow \texttt{Num(11)}} \\
\texttt{Not(Num(11))} \Rightarrow \texttt{Boolean(false)} \\
\end{array}
\]

\textbf{Resultado Final:} \(\texttt{Boolean(false)}\)

\subsubsection*{(c) Evaluación con Semántica Estructural}

\[
\begin{array}{c}
\text{La expresion es: } \texttt{Not(Add(Num(1), Sub(Num(3), Add(Num(-8), Num(1)))))}. \\
\\
\Rightarrow \texttt{Not(Add(Num(1), Sub(Num(3), Add(Num(-8), Num(1)))))} \\
\Rightarrow \texttt{Not(Add(Num(1), Sub(Num(3), Num(-7))))} \\
\Rightarrow \texttt{Not(Add(Num(1), Num(10)))} \\
\Rightarrow \texttt{Not(Num(11))} \\
\Rightarrow \texttt{Boolean(false)} \\

\end{array}
\]

\textbf{Resultado Final:} \(\texttt{Boolean(false)}\)

\subsection*{Expresión 3: \texttt{(not (not (+ 3 5)))}}

\subsubsection*{(a) Sintaxis Abstracta}
\[
\text{Sintaxis concreta: } \texttt{(not (not (+ 3 5)))}
\]
\[
\text{Sintaxis abstracta: } \texttt{Not(Not(Add(Num(3), Num(5))))}
\]

\subsubsection*{(b) Evaluación con Semántica Natural}

Evaluamos \(\texttt{Add(Num(3), Num(5))}\):
\[
\begin{array}{c}
\\
\underline{\texttt{Num(3)} \Rightarrow \texttt{Num(3)}}
\underline{\texttt{Num(5)} \Rightarrow \texttt{Num(5)}} \\
\texttt{Add(Num(3), Num(5))} \Rightarrow \texttt{Num(8)}
\end{array}
\]

Evaluamos \(\texttt{Not(Num(8))}\):
\[
\begin{array}{c}
\\
\underline{\texttt{Num(8)} \Rightarrow \texttt{Num(8)}} \\
\texttt{Not(Num(8))} \Rightarrow \texttt{Boolean(false)}
\end{array}
\]

Evaluamos \(\texttt{Add(Num(1), Num(10))}\):
\[
\begin{array}{c}
\\
\underline{\texttt{Num(1)} \Rightarrow \texttt{Num(1)}}
\underline{\texttt{Num(10)} \Rightarrow \texttt{Num(10)}} \\
\texttt{Add(Num(1), Num(10))} \Rightarrow \texttt{Num(11)}
\end{array}
\]

Evaluamos \(\texttt{Not(Boolean(false))}\):
\[
\begin{array}{c}
\\
\underline{\texttt{Boolean(false)} \Rightarrow \texttt{Boolean(false)}} \\
\texttt{Not(Boolean(false))} \Rightarrow \texttt{Boolean(true)} \\
\end{array}
\]

\textbf{Resultado Final:} \(\texttt{Boolean(true)}\)

\subsubsection*{(c) Evaluación con Semántica Estructural}

\[
\begin{array}{c}
\text{La expresion es: } \texttt{Not(Not(Add(Num(3), Num(5))))}. \\
\\
\Rightarrow \texttt{Not(Not(Add(Num(3), Num(5))))} \\
\Rightarrow \texttt{Not(Not((Num(8)))} \\
\Rightarrow \texttt{Not(Boolean(false))} \\
\Rightarrow \texttt{Boolean(true)} \\

\end{array}
\]

\textbf{Resultado Final:} \(\texttt{Boolean(true)}\)

\hspace{1 cm}Como segundo ejercicio deberán extener la batería de operaciones de MiniLisp, para ello deberán (a) dar la gramática libre de contexto modificada (en notación EBNF) añadiendo las nuevas construcciones del lenguaje, (b) modificar las reglas de sintaxis abstracta para considerar los nuevos constructores y finalmente (c) extender las reglas de semántica natural y estructural. En los tres casos, deberás usar la notación formal que vimos en clase.
\vspace{1 cm} \\ Grática Modificada en EBNF (aquello que no se modifico, permanece igual que en la gramática ya propuesta.
\\ <Expr> := <Int> \\ . \hspace{1.5 cm} | <boolean> \\ . \hspace{1.5 cm} | ( + <Expr>  \space <Expr> ) \\ . \hspace{1.5 cm} | ( - <Expr> \space <Expr> ) \\ . \hspace{1.5 cm} | not (<Expr>)\\ . \hspace{1.5 cm} | ( * <Expr>  \space <Expr> ) \\ . \hspace{1.5 cm} | ( / <Expr> \space  <Expr> ) \\ . \hspace{1.5 cm} | ( Add1 <Expr> ) \\ . \hspace{1.5 cm} | (sub1 <Expr> ) \\ . \hspace{1.5 cm} | ( sqrt <Expr> ) 

 \vspace{1 cm}Reglas de sintaxis abstracta

 \begin{prooftree}
     \AxiomC{i ASA}
     \AxiomC{d ASA}
     \BinaryInfC{mult(i,d) ASA}
     \end{prooftree}

 \begin{prooftree}
     \AxiomC{i ASA}
     \AxiomC{d ASA}
     \BinaryInfC{div(i,d) ASA}
     \end{prooftree}

 \begin{prooftree}
     \AxiomC{e ASA}
     \UnaryInfC{sqrt(e) ASA}
     \end{prooftree}

     \begin{prooftree}
     \AxiomC{e ASA}
     \UnaryInfC{add1(e) ASA}
     \end{prooftree}

      \begin{prooftree}
     \AxiomC{e ASA}
     \UnaryInfC{sub1(e) ASA}
     \end{prooftree}

     \vspace{1 cm} Semántica Natural 
     \\  
     \begin{prooftree}
     \AxiomC{i $\Longrightarrow$ num(i')}
     \AxiomC{d $\Longrightarrow$ num(d')}
     \BinaryInfC{mult(i,d) $\Longrightarrow$ num(i' * d')}
     \end{prooftree}

  \begin{prooftree}
     \AxiomC{i $\Longrightarrow$ num(i')}
     \AxiomC{d $\Longrightarrow$ num(d'); d´ $\not=$ 0}
     \BinaryInfC{div(i,d) $\Longrightarrow$ num(i' / d')}
     \end{prooftree}

     
     \begin{prooftree}
     \AxiomC{e $\Longrightarrow$ num(e')}
     \UnaryInfC{add1(e) $\Longrightarrow$ num(e' +1)}
     \end{prooftree}

     \begin{prooftree}
     \AxiomC{e $\Longrightarrow$ num(e')}
     \UnaryInfC{sub1(e) $\Longrightarrow$ num(e' - 1)}
     \end{prooftree}

     \begin{prooftree}
     \AxiomC{e $\Longrightarrow$ num(e') ; 0 $\leq$ e'}
     \UnaryInfC{sqrt(e) $\Longrightarrow$ sqrt(e')}
     \end{prooftree}

     Semántica Estructural

a) Multiplicación
      \begin{prooftree}
     \AxiomC{i $\longrightarrow$ i'}
     \UnaryInfC{mult(i,num(d)) $\longrightarrow$ mult(i', d)}
     \end{prooftree}

       \begin{prooftree}
     \AxiomC{d $\longrightarrow$ d'}
     \UnaryInfC{mult(num(i),d) $\longrightarrow$ mult(i, d')}
     \end{prooftree}

       \begin{prooftree}
       \AxiomC{}
     \UnaryInfC{mult(num(i),num(d)) $\longrightarrow$ num(i * d)}
     \end{prooftree}
     
b) División

  \begin{prooftree}
     \AxiomC{i $\longrightarrow$ i'}
     \UnaryInfC{div(i,num(d) ; d $\not= 0$) $\longrightarrow$ div(i',d)}
     \end{prooftree}

     \begin{prooftree}
     \AxiomC{d $\longrightarrow$ d'}
     \UnaryInfC{div(num(i),d) $\longrightarrow$ div(i, d')}
     \end{prooftree}

     \begin{prooftree}
     \AxiomC{}
     \UnaryInfC{div(num(i),num(d); d $\not= 0$) $\longrightarrow$ num(i / d)}
     \end{prooftree}

c) Add1     
     \begin{prooftree}
     \AxiomC{}
     \UnaryInfC{add1(num(e)) $\longrightarrow$ num(e' +1)}
     \end{prooftree}

     \begin{prooftree}
     \AxiomC{e $\longrightarrow$ e'}
     \UnaryInfC{add1(e) $\longrightarrow$ add1(e')}
     \end{prooftree}

d) Sub1
     \begin{prooftree}
     \AxiomC{}
     \UnaryInfC{sub1(num(e)) $\longrightarrow$ num(e' - 1)}
     \end{prooftree}
     
     \begin{prooftree}
     \AxiomC{e $\longrightarrow$ e'}
     \UnaryInfC{sub1(e) $\longrightarrow$ sub1(e')}
     \end{prooftree}

e) Raiz
    
    \begin{prooftree}
     \AxiomC{}
     \UnaryInfC{sqrt(num(e) ; 0 $\leq e$) $\longrightarrow$ sqrt(e')}
     \end{prooftree}

     \begin{prooftree}
     \AxiomC{e $\longrightarrow$ e'}
     \UnaryInfC{sqrt(e) $\longrightarrow$ sqrt(e')}
     \end{prooftree}
\end{document}  